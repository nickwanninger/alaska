\documentclass[sigplan,screen]{acmart}


\usepackage{lipsum}  



\begin{document}

\title{Alaska: Handle-Based Memory \\ Management Made Easy}


\author{Nick Wanninger}
\affiliation{
  \institution{Northwestern University}
  \city{Evanston}
  \country{United States}}
\email{ncw@u.northwestern.edu}

\author{Peter Dinda}
\affiliation{
  \institution{Northwestern University}
  \city{Evanston}
  \country{United States}}
\email{pdinda@northwestern.edu}

\newcommand{\hbmm}{handle-based memory management~}

\begin{abstract}
Prior to hardware virtual memory multitasking operating systems placed all
programs in the same address space, potentially leading to significant memory
fragmentation. To solve this problem, many systems turned to handles: a level
of abstraction on top of pointers which allow the operating system to relocate
memory without invalidating a program's pointers. To use handles, programmers
were required to manually "Lock" and "Unlock" these pointers to allow the
operating system relocate and defragment memory without the worry of corrupting
the program. Unfortunately, this led to many of the same bugs found in manual
memory management such as stale references or memory leaks.

In this paper, we present Alaska - a compiler and run-time system which allows
programmers to use \hbmm on unmanaged languages without changes to the source
code. Using modern compiler techniques to insert the lock and unlock run-time
calls, applications can transparently utilize features like defragmentation,
compression, and copy-and-update semantics -- effectively allowing virtual
memory to operate at the granularity of individual allocations. We show that
Alaska does all this with minimal overhead to applications which don't use
handles, but provides a great deal of control to programs that do.
\footnote{A guess based on the naive compiler}


\end{abstract}


\maketitle


\section{Introduction}
\lipsum[2-4]

\section{Handles}
\lipsum[2-4]

\section{Design}
\lipsum[2-4]
\subsection{Compiler}
\lipsum[2-4]
\subsection{Run-times}
\lipsum[2-4]

\section{Evaluation}
\lipsum[2-4]

\section{Related Work}
\lipsum[1-2]

\section{Conclusion}
\lipsum[1-2]
 
\end{document}
\endinput
%%
%% End of file `sample-sigplan.tex'.
